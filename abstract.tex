地理的犯罪予測の代表的手法であるRisk Terrain Modeling(RTM)は,廃屋や廃棄車両などの位置情報から作成した特徴量から,近い将来における犯罪発生リスクを算出する手法である.しかしながら既存手法による予測は実際の犯罪発生の空間分布とは乖離した高い空間相関をもち,不十分な予測精度として現れていた.この観察に基づき,本研究では既存手法において離散変数として取り扱われていた特徴量の改善を行った.この改善により,犯罪予測と機械学習の文脈で一般的であるPAIとAUCの意味において,従来手法をそれぞれ62.7%,13.5%改善した.
