%------------------------------------
\subsection{離散型特徴量(Discrete features, DF)}
%------------------------------------
従来手法\citep{caplan2015risk}による,距離とカーネル密度推定値による離散型特徴量を
予測変数とするモデルをDFとする.
%------------------------------------
\subsection{連続型特徴量(Continuous features, CF)}
%------------------------------------
DFでは,距離とカーネル密度推定値による連続型特徴量を離散化するが,
CFでは離散化せず連続型特徴量のままRTMの予測変数とする.

%------------------------------------
\subsection{負の冪での変換(Negative exponent, NE)}
%------------------------------------
NEでは,CFの特徴量の連続化の発展として,距離特徴量$d$を負の冪(\ref{inverse})式で変換する.

\begin{equation}\label{inverse}
  \abs{d}^{-\gamma}
\end{equation}

ここで,$d$は距離,$\gamma$は指数を決定する正のパラメータである.

%------------------------------------
\subsection{ラプラス分布関数での変換(Laplace distribution, LD)}
%------------------------------------
LDでは,CFでの特徴量の連続化の発展として,距離特徴量$d$を
規格化定数を除いたラプラス分布関数(\ref{laplace})式で変換する.

\begin{equation}\label{laplace}
  \exp \left( -\frac{\abs{d}}{\sigma}\right)
\end{equation}

ここで,$d$は距離,$\sigma$はラプラス分布の尺度を決定するパラメータである.


%------------------------------------
\subsection{ガウス関数での変換(Gauss function,GF)}
%------------------------------------
GFでは,CFでの特徴量の連続化の発展として,距離特徴量$d$をガウス関数(\ref{gauss})式で変換する.

\begin{equation}\label{gauss}
  \exp \left(-\frac{d^2}{2\sigma^2} \right)
\end{equation}

ここで,$d$は距離,$\sigma$はガウス関数の尺度を決定するパラメータである.

% ------------------------------------
\subsection{2次元ガウス特徴量(Two-dimensional Gaussian feature, TG)}
%------------------------------------
TGでは,CFでの特徴量の連続化の発展として,各グリッドセル中心点$l$に対して,
kNNにより取得した \( K \) 個のある地理的リスク要因の近傍点から2次元ガウス分布を求め、
その確率密度関数 (PDF) を表現する.
各グリッドセル中心点 \( l \) に対して 
kNN を用いて \( K \) 個の近傍点 \( \{ x_1, x_2, \dots, x_K \} \) を取得する.
各点 \( x_i \) は2次元座標 \( x_i = (x_{i1}, x_{i2})^\top \) を持つとする.

近傍点の平均ベクトル \( \mu \) を(\ref{TG-mu})式から求める.

\begin{equation}\label{TG-mu}
  \mu = \frac{1}{K} \sum_{i=1}^{K} x_i
\end{equation}

次に共分散行列 \( \Sigma \) を(\ref{TG-Sigma})式で求める
\begin{equation}\label{TG-Sigma}
  \Sigma = \frac{1}{K} \sum_{i=1}^{K} (x_i - \mu)(x_i - \mu)^\top
\end{equation}

各グリッドセル中心点 \( l \) の座標を \( x_l \) とすると,
その地点におけるガウス分布の値は(\ref{TG-pdf})式のように表される.

\begin{equation}\label{TG-pdf}
p(x_l) = \frac{1}{2\pi |\Sigma|^{1/2}} \exp \left( -\frac{1}{2} (x_l - \mu)^\top \Sigma^{-1} (x_l - \mu) \right)
\end{equation}

この値を2次元ガウス特徴量として利用することで,各グリッドセル中心点 \( l \)での
局所的な地理的リスク要因の分布情報をモデルに組み込む.

% ------------------------------------
\subsection{ガウス関数での変換と2次元ガウス特徴量(GF+TG)}
%------------------------------------
距離特徴量をガウス関数によって変換するGFと,
2次元ガウス特徴量を新たに生成するTGを組み合わせたモデルをGF+TGとする.