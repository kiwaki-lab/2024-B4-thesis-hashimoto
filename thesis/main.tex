%------------------------------------
%   basic settings
%------------------------------------
\documentclass[12pt,a4paper,oneside]{jsbook}
\usepackage[T1]{fontenc}
\usepackage{lmodern}
\usepackage{amsmath}
\usepackage{amsthm}
\usepackage{amssymb}
\usepackage[dvipdfmx]{graphicx}
\usepackage{url}
\usepackage{here}
\usepackage{algorithm}
\usepackage[noend]{algpseudocode}
\usepackage[ipaex]{pxchfon}
\usepackage{otf}
\usepackage{listings}
%------------------------------------
%   listings settings (minted -> listings)
%------------------------------------
\lstset{
  basicstyle=\ttfamily\small,  % Font style
  numbers=left,                % Add line numbers
  numberstyle=\tiny,           % Line number style
  stepnumber=1,                % Line number increment
  frame=single,                % Add a frame around the code
  tabsize=4,                   % Tab size
  breaklines=true,             % Allow line breaking
  keywordstyle=\bfseries,      % Keywords in bold
  commentstyle=\itshape,       % Comments in italics
  stringstyle=\color{red},     % Strings in red
  showspaces=false,            % Do not mark spaces
  showstringspaces=false,      % Do not mark string spaces
  language=Python              % Default language
}

%------------------------------------
%   margin settings
%------------------------------------
\setlength{\topmargin}{-5mm}
\setlength{\fullwidth}{125mm}
\setlength{\textwidth}{\fullwidth}
\setlength{\oddsidemargin}{5mm}
\setlength{\evensidemargin}{\oddsidemargin}
%------------------------------------
%   newtheorems
%------------------------------------
\theoremstyle{plain}
\newtheorem{theorem}{定理}[chapter]
\newtheorem{corollary}[theorem]{系}
\newtheorem{lemma}[theorem]{補題}
\newtheorem{fact}[theorem]{Fact}
\newtheorem{conjecture}[theorem]{予想}
\newtheorem{proposition}[theorem]{命題}
\newtheorem{problem}[theorem]{問題}
\newtheorem{definition}[theorem]{定義}
\newtheorem{remark}[theorem]{Remark}
\newtheorem{claim}{Claim}
\newtheorem{subclaim}{Subclaim}[claim]
\newcommand{\resetclaim}{\setcounter{claim}{0}}
\newtheorem{case}{Case}
\newtheorem{subcase}{Subcase}[case]
\newcommand{\resetcase}{\setcounter{case}{0}}
%------------------------------------
%   display figures
%   #1=width, #2=filename,
%   #3=caption, #4=label
%   \fig{0.8\linewidth}{aaa.pdf}{bbb}{ccc}
%------------------------------------
\renewcommand{\figurename}{図.}
\newcommand{\fig}[4]{
\begin{figure}[H]
\centering
\includegraphics[width=#1]{#2}
\caption{#3}
\label{#4}
\end{figure}
}
\newcommand{\figg}[4]{
\begin{figure*}[h!t]
\centering
\includegraphics[width=#1]{#2}
\caption{#3}
\label{#4}
\end{figure*}
}
%------------------------------------
%   setting of algorithms
%------------------------------------
\renewcommand{\algorithmicrequire}{\textbf{条件:}}
\renewcommand{\algorithmicensure}{\textbf{実行結果:}}
\algrenewcommand\algorithmicdo{}
\algrenewcommand\algorithmicthen{}
%------------------------------------
%   other renewcommands and newcommands
%------------------------------------
\renewcommand{\proofname}{\bf 証明.}
%------------------------------------
%   Title & Authors
%------------------------------------
\title{
卒業論文\\[1.5cm]
特徴量連続化によるRisk Terrain Modelingの改良\\[6cm]
}
\author{高知大学 理工学部 情報科学科\\[0.5cm]
B213R018Y 橋本響}
\date{2024年度}

%------------------------------------
\begin{document}
%------------------------------------
%タイトルページの出力
\maketitle
%目次の作成・出力
\tableofcontents

%------------------------------------
%   Chapter 1
\chapter{はじめに}
\label{chapter_1}
%------------------------------------
\section{背景}
%------------------------------------
犯罪の防止は古今を問わず重要な社会課題である.
近年では地理情報システムの発展に伴い,犯罪が発生した時間・場所などの各種データが蓄積されており,
それらに基づく犯罪予測は新たな政策提案・警察活動の指針を与えるものとして国際的に期待されている.
その代表的手法である
Caplan et al.(2011)\cite{caplan2011}
が提唱するRisk Terrain Modeling(RTM)は,廃屋や廃棄車両などの位置情報から作成した特徴量から,
近い将来における犯罪発生リスクを算出し,視覚化する手法として広く利用されている.
RTMにより,政策立案者や治安維持機関は,リスクの高いエリアを特定し,
資源の効率的な配分に役立てる事ができる.

%------------------------------------
\section{従来手法の課題}
%------------------------------------
従来のRTM手法では,予測変数として地理的な特徴量を離散的なカテゴリデータとして扱うことが一般的であり,
基盤となる統計モデルは主に線形モデルとして構築されてきた.
しかし,従来手法の予測結果は高い空間相関を持つという課題がある.
この振る舞いは実際の犯罪発生の空間分布とは乖離しており,不十分な予測精度として現れている.

%------------------------------------
\section{研究目的}
%------------------------------------
本研究では,RTMにおける特徴量エンジニアリングの手法を拡張し,
従来手法が抱える課題に対処することを目的とする.
具体的には,以下の2つのアプローチを提案する.

1つ目は,従来離散的に扱われていた地理的リスク要因を連続的にモデリングすることで,
情報の損失を抑え,より正確なリスク評価を可能にする. 
2つ目は,距離に関連するリスク要因に対して,ガウス関数を用いた非線形変換を適用することにより,
空間的影響の変化をより現実的に反映する.
これらの改良により,モデルの予測精度を向上させるとともに,空間相関を削減することを目指す.
%------------------------------------
\chapter{関連研究}
\label{chapter_2}
%------------------------------------
\section{RTMの現状}
%------------------------------------
従来のRTMでは,地理的リスク要因の影響を主に離散的なカテゴリデータとして表現してきた.
例えば,特定の地理的リスク要因(廃屋,放置車両など)
から一定距離内に存在するか否かを0または1で表す手法が一般的であった
Caplan et al.(2015)\cite{caplan2015}
この方法は実際の空間的影響を十分に表現できない可能性がある.
%------------------------------------
\section{予測精度向上への取り組み}
%------------------------------------

%------------------------------------
\chapter{研究手法}
\label{chapter_3}
%------------------------------------
本研究では,
Caplan et al.(2015)\cite{caplan2015}
が提唱するRTMと同様に,
アメリカ合衆国のイリノイ州クック郡の郡庁所在地であるシカゴ市を対象領域とする.
RTMの予測精度を向上させるために,2つの新しい特徴量構成手法を提案する.
既存研究の発想に基づいて離散型特徴量を用いた予測モデル(モデル0),
本研究で提案する特徴量を連続化する手法による予測モデル(モデル1),
本研究で提案する距離特徴量をガウス関数で変換する手法による予測モデル(モデル2),
の3種類を実装し,それぞれの予測精度を比較する.
以下では,RTMの構成と各手法の詳細について説明する.

%------------------------------------
\section{RTMの構成}
%------------------------------------
分析にあたっては,対象地域に130m×130mのグリッドセルを設定し,各グリッドセルでの犯罪発生リスクを予測する.
応答変数は各グリッドセルで発生した強盗犯罪件数,予測変数は地理的リスク要因から生成した特徴量である.
予測変数は,ユークリッド距離による特徴量とカーネル密度推定値による特徴量の2種類の特徴量から構成される.

モデルの構成方法は,
まず,データの前処理として予測変数を,PowerTransformで正規分布に近づけ,この処理に加えて標準化する.
次に,20fold交差検証でLasso回帰の適切なペナルティパラメータを探索し変数選択する.
最後に,予測値に対してPowerTransformの逆変換を行い,元の分布に近づけて犯罪発生リスクを算出する.
%------------------------------------
\section{従来手法による特徴量の離散化(モデル0)}
%------------------------------------
モデル0では,
Caplan et al.(2015)\cite{caplan2015}
が提唱する特徴量構成方法に従って,離散型特徴量を構成する.
距離に関する特徴量は,各グリッドセル中心点からそれぞれの地理的要因までの最短距離を
2水準(0:特定の距離外,1:特定の距離内)に離散化する.
また,カーネル密度推定に関する特徴量は,
それぞれの地理的要因について特定のバンド幅でカーネル密度推定を行い,
各グリッドセル中心点での密度推定値によって,
2水準(0:平均+2標準偏差未満,1:平均+2標準偏差以上)に離散化する.

なお,特定の距離とバンド幅とは271m,591m,779m,1003mである.
これらの距離はシカゴ市の2ブロック,4ブロック,6ブロック,8ブロックに相当する.
%------------------------------------
\section{提案手法による特徴量の連続化(モデル1)}
%------------------------------------
モデル1では,
Caplan et al.(2015)\cite{caplan2015}.
による従来手法では距離とカーネル密度推定値による連続型特徴量を離散化しているが,
離散化せず連続型特徴量のままRTMの予測変数とする.
\section{提案手法による他犯罪特徴量の追加(モデル1)}
\section{提案手法による応答変数の正規化(モデル1)}
\section{提案手法によるラプラス分布関数での変換(モデル2)}
%------------------------------------
\section{提案手法によるガウス関数での変換(モデル2)}
%------------------------------------
モデル2では,モデル1での特徴量の連続化の発展として,距離特徴量をガウス関数(\ref{gauss})式で変換する.

\begin{equation}\label{gauss}
  \exp(-\frac{d^2}{2\sigma^2})
\end{equation}

ここで,$d$は距離,$\sigma$はガウス分布の尺度を決定するパラメータである.
$log$スケールで5種類の値の$\sigma$を用いて,1つの地理的リスク要因の距離特徴量から,
ガウス関数で変換した5つの特徴量を生成した.
このアプローチは,最短距離が大きくなれば犯罪への影響度は小さくなるという直感的な解釈に基づいて導入する.

%------------------------------------
\section{RTMの実装}
%------------------------------------
%------------------------------------
\chapter{実験}
\label{chapter_4}
%------------------------------------
\section{データセット}
%------------------------------------
Chicago Data Portalからシカゴ市の2011年から2014年の強盗犯罪発生地点および,
地理的要因(例:廃屋,放置車両など)の緯度経度情報を取得した.

取得したデータは,地理的座標系に基づいており,緯度経度情報として提供されているが,
本研究では距離計算や空間分析を正確に行うため,メートル単位での解析が可能な平面直角座標系に変換した.
座標系の変換には,pythonのgeopandasライブラリを使用し,
犯罪発生地点および地理的要因の位置情報をすべて統一した基準で処理した.
また,データ品質を保証するため,明らかに誤った位置情報は事前に除外した.
%------------------------------------
\section{実験条件}
%------------------------------------
まず,2011〜2012年を学習データ,2013年を検証データとして,最適な予測モデルを探索し,
最終的に,2011〜2013年を学習データ,2014年をテストデータとして,3種のモデルの予測精度を比較する.

予測変数として用いた地理的リスク要因は,
廃屋,放置車両,木柱街灯の消灯,金属柱街灯の消灯,学校,差し押さえ物件,落書き,不衛生な場所
の8つである.
%------------------------------------
\section{評価方法}
%------------------------------------
モデルが予測した犯罪発生リスクを,
高リスク(平均+1標準偏差以上)と低リスク(平均+1標準偏差未満)にカテゴリー化する.
犯罪予測の文脈で一般的である的中率とPAIの2つの指標で,年単位で予測精度を評価する.
的中率とは,

%------------------------------------
\chapter{結果}
%------------------------------------
%------------------------------------
%   Chapter 5
\chapter{おわりに}
\label{chapter_5}
%------------------------------------
これまでに書いた内容を簡潔にまとめる。
ただし,第1章とは異なり,
ここまで論文を読み終えた読者を想定しているため,
これまでに登場した専門用語や事実等を
改めて説明することなく使用してかまわない。
「~を考察した。」ではなく,
「~であることが分かった。」のように
結論を述べる。
今後の課題も書く。


%------------------------------------
%   Acknowledgements
\chapter*{謝辞}
%------------------------------------
\addcontentsline{toc}{chapter}{謝辞}
(研究を遂行するにあたってサポートしてくれた方々へ
謝辞を述べるコーナー。
名前だけでなくどのようなサポートをしてくれたのかも書く。)

(例)
本研究を進めるにあたりご指導頂きました
鈴木一弘先生に感謝いたします。
本論文で主査と副査をして頂きました
\CID{8705}田直樹先生と塩田研一先生に感謝いたします。
日頃の議論を通じて多くの知識や示唆を頂きました
鈴木研究室の皆様に感謝いたします。



%------------------------------------
%   References
%------------------------------------
\renewcommand{\bibname}{参考文献}
\begin{thebibliography}{99}


\bibitem{caplan2015}
Caplan, Joel;
Kennedy, Leslie;
Barnum, Jeremy
Piza, Eric
{Risk Terrain Modeling for Spatial Risk Assessment.}.
{\em Cityscape}
{\bfseries 17(1)}
(2015).
7-16



\end{thebibliography}


%------------------------------------
\appendix
%------------------------------------
\chapter{プログラム}
\label{program}
% \inputminted[breaklines,breakanywhere,linenos,tabsize=4]
% {JavaScript}{discrete_geometry.js}


%------------------------------------
\end{document}
%------------------------------------
